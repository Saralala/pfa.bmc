Lors du développement de ce projet, l'accessibilité était un argument essentiel à prendre compte dans les choix d'implémentation. Ainsi, nous n'avons pas élu la bibliothèque graphique \textit{Qt}, car bien qu'adaptée à plusieurs systèmes d'exploitation, elle risquait de poser certains problèmes de compatibilité pour des utilisateurs non-voyant se servant de lecteurs d'écran.  
%L'un des critères essentiels de ce projet est l'accessibilité. Ce qui fait que nous n'avons pas pu utilisé une bibliothèque graphique cross-platform comme Qt, car ce type de bibliothèque a des prolèmes de compatibilité avec les lecteurs d'écran des systèmes d'exploitation. 
Avec les suggestions du M. Lang et celles de M.Thibaut, les bibliothèques graphiques \textit{GTK+} sous Linux et \textit{Winforms} sous Windows ont ainsi été privilégiées.

\subsection{Unix}
% unix.tex


\subsection{Windows}
% windows.tex



