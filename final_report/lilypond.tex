% lilypond.tex

% ben : je veux bien m'en charger
\subsubsection{Génération du fichier Lilypond}
L'exportation vers le \textit{Lilypond} a été traitée en premier lieu.
Ce format est destiné à produire des partitions au format PDF de
grande qualité. Ce choix est essentiel, car d'une part le
format PDF est très répandu, et d'une autre cela a permis d'afficher la partition pour voyant dans l'interface graphique.

L'implémentation de cette fonctionnalité consiste en la création de la
classe \textit{toLily}. Une partie des méthodes de cette classe vont
parcourir le langage Abstrait (la structure \textit{score}) produit par le \textit{BMC}
pour le convertir en langage compréhensible par \textit{Lilypond}. Chaque
objet \textit{struct sign} contenu dans la structure \textit{score} est
associé à une méthode qui traitera celui-ci et le transformera en un
caractère compris par \textit{Lilypond}.

\noindent
La Figure \ref{resultatlilypond} montre un exemple de fichier \textit{Lilypond} produit par \textit{BMC}
ainsi que le résultat produit par \textit{Lilypond}.

\begin{verbatim}
% Automatically generated by BMC, the braille music compiler
\version "2.14.2" 
\score {
  <<
    \new PianoStaff <<
      \new Staff {
        \clef "treble"
        \key g \major
        \partial 8 
        << {r8}\\{r8} >> | % 0
        << {r2 r4 r8 d'8-1}\\{r1} >> | % 1
        << {g'8-2 a'8-1 b'8-3 c''8 d''8 c''16 b'16 a'8-2 r8}\\
           {g'4-2 g'4-2 a'4-1 a'4} >> | % 2   
        << {d''4-5 d''4 e''4 e''4-5}\\
           {g'8-1 a'8 g'8 f'8-2 e'8 d'8  c'8  b'8-1} >>| % 3
        <d''-4 g''-2>8 <e'' c'''>8 <d'' g''>8 <c''-5 g''-1>8 
           << {b'8 a'8 g'4}\\{g'4 g'8 d'8} >> | % 4
      }
      \new Staff {
        \clef "bass"
        \key g \major
        \partial 8 
        << {r8}\\{d8-5} >> | % 0
        << {g2-2 fis2}\\{g8-2  a8-1 b8 c'8 d'8 c'16 b16  a8-1 g16 a16} >> | % 1
        << {e2 d4.-4 c8}\\{b8-2 a8 g2-1 fis8 e16 fis16} >> | % 2
        << {b,2 c2}\\{g4 r8 d'8 c'8 b8 a8 g8-1} >> | % 3
        << {d2 g,4 r8 b16 a16}\\{fis4~ fis16 e32 fis32 d8~-1 d4 r4} >> | % 4
      }
    >>
  >>

  \layout { }
  \midi { }
}
\end{verbatim}

\begin{figure}[!h]
  \centering
  \includegraphics[width=1\textwidth]{images/lilypond_exe.png}
  \caption{Musique produit par \textit{Lilypond}}
  \label{resultatlilypond}
\end{figure}


Cette classe a été retouchée par M. Lang pour pouvoir l'intégrer dans \textit{BMC}.
%Une fois cette classe implémenté nous l'avons partagé avec M Lang. Il
%l'a ensuite amélioré avant de l'intégrer dans bmc.


%% \subsubsection{Génération du format pdf}
%% Une fois le fichier lilypond généré par cette nouvelle version de BMC,
%% il est aisé d'obtenir un pdf de la partition correspondante à l'aide
%% de la commande : $$lilypond\ fichier.ly$$.
%Cependant nous avions besoin dans notre interface graphique de
%récupérer certaines informations lorsque l'utilisateur clique sur les
%notes de ce pdf.


\subsubsection{Amélioration du pdf produit par Lilypond}

L'IDE a pour but de faire travailler un voyant avec un aveugle,
pour faciliter le travail colaboratif. Il paraît donc judicieux d'ajouter
une fonctionnalité à la fois utile et agréable : pointer une
note sur le PDF amène le curseur à cette note dans le fichier
braille.



Lors de la construction du PDF, le programme \textit{Lilypond} se
charge de rajouter des liens entre les notes musicales résultantes et
le code correspondant dans le fichier \textit{.ly}.  En exploitant
cette propriété, un programme a été mis en place pour établir une
correspondance entre les notes du PDF est celles écrites en braille
musical.  

Ce programme réalise trois étapes. Il commence par ajouter en
commentaire, après chaque note dans le fichier \textit{.ly}, les
informations récupérées depuis \textit{BMC} (les lignes et les colonnes des
notes dans le fichier braille) puis à compiler ce fichier en
format \textit{ps} grâce à la commande $lilypond\ --ps\
fichier.ly$. \textit{Lilypond} fait déjà la correspondance des notes
dans le fichier PDF et leur emplacement dans le fichier \textit{lilypond}.  La
deuxième étape consiste à supprimer cette correspondance avec le
fichier \textit{lilypond} et ajouter une correspondance avec le fichier
braille. Pour cela, il remplace tous les liens vers le
fichier \textit{lilypond} par des liens vers le fichier braille (la
position de la note dans le fichier braille étant lu dans le fichier
\textit{lilypond}). Enfin, la dernière étape consiste à transformer le
fichier \textit{.ps} en fichier \textit{.pdf} ; pour cela on utilise un
package Linux : ps2pdf.

%liens par les informations souhaitées ,\textit{i.e.} la position de
%la note dans le ficheir braille music. Après avoir rajouté en
%commentaire ces informations après chaque note dans le fichier
%lilypond, ce programme compile d'abord le fichier en format ps grace
%à la commande $$lilypond\ --ps\ fichier.ly$$ Puis il remplace tous
%les liens vèrs le fichier lilypond par la position de la note dans le
%fichier braille music, enfin il compile le ps obtenue en pdf grace à
%la commade : $$ps2pdf\ fichier.ps$$

Le PDF obtenu est ensuite affiché dans l'interface graphique qui
exploite les liens vers le braille musical pour positionner le curseur
à l'endroit correspondant.
