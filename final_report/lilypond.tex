% lilypond.tex

% ben : je veux bien m'en charger
\subsubsection{Génération du fichier Lilypond}
Nous nous somme dans un premier temps consacré à l'exportation sous le
format lilypond.  Ce format est destiné à produire des partitions au
format pdf de grande qualité.  Ce choix nous est apparut fondamental,
d'une part le format pdf est très répandu, d'une autre cela nous a
permit de pouvoir afficher la partition pour voyant dans notre
interface graphique.

L'implémentation de cette fonctionalité consiste en la création de la
classe \textit{toLily}. Les fonctions de cette classe parsent le
langage abstrait produit par le \textit{BMC} pour le convertir en langage
lilypond. Ce dernier ressemble un peu au langage \LaTeX. À chaque
structure contenue dans le premier langage on associe une action qui
écrira dans le fichier lilypond généré et mettra à jour les variables
de la classe.

Une fois cette classe implémenté nous l'avons partagé avec M Lang. Il
l'a ensuite amélioré avant de l'intégrer dans \textit{BMC}.


\subsubsection{Génération du format pdf}
Une fois le fichier lilypond généré par cette nouvelle version de \textit{BMC}
crée, il est facile d'obtenir un pdf de la partition correspondante à
l'aide de la commande : $$lilypond\ fichier.ly$$ Cependant nous
avions besoin dans notre interface graphique de récupérer certaines
informations lorsque l'utilisateur clique sur les notes de ce pdf.

Lors de la construction du pdf le logiciel lilypond ce charge de
rajouter des liens des notes vèrs la note correspondante dans le
fichier \textit{.ly}. Nous avons crée un programme pour remplacer ces
liens par les informations souhaitées ,\textit{i.e.} la position de la
note dans le fichier \textit{braille music}. Après avoir rajouté en commentaire
ces informations après chaque note dans le fichier lilypond, ce
programme compile d'abord le fichier en format ps grace à la commande
$$lilypond\ --ps\ fichier.ly$$ Puis il remplace tous les liens
vèrs le fichier lilypond par la position de la note dans le fichier
\textit{braille music}, enfin il compile le ps obtenue en pdf grace à la
commade : $$ps2pdf\ fichier.ps$$

Le pdf obtenu est ensuite utilisable par notre interface graphique.
