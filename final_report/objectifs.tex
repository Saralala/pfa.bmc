% objectifs.tex

Bien que la majorité des objectifs de ce projet ait été atteint il
reste quelques fonctionnalités non implémentées. Celles-ci n'étant
que secondaires cela ne pose pas de problème majeur à notre client au
vu du temps dont disposait notre équipe. En effet celui-ci désirait
que soit faites en priorité les tâches qui lui poseraient le plus de
problèmes.

Voici un résumé de ces fonctionnalités et quelques pistes de
développement.

 
\subsection{Portage sous Windows}
L'interface graphique de \textit{BMC} sous Windows étant implémentée
il reste à faire le portage du programme. Ce portage n'était
initialement pas contenu dans les besoins du client. Celui-ci nous a
proposé cette tâche ultérieurement. Cependant ce travail nécessite une
dépense en temps trop importante car il implique de s'intéresser à
tout le code du \textit{BMC} y compris au \textit{frontend} qui n'a jamais
été étudié.


\subsection{Portage sous MacOs}
L'implémentation de l'\textit{IDE} sous MacOs est envisagée car il
reste un des systèmes les plus utilisés. Cependant, à l'heure
actuelle, le pourcentage d'utilisateurs de ce système est trop faible
pour permettre une dépense de temps importante à re-coder une
\textit{GUI}.

\subsection{Autres formats de sortie pour BMC}
La possibilité de créer des sorties dans d'autres formats (ex: avi,
mp3, ...) est envisageable. Pour cela il existe deux moyens
d'implémentation. Le premier consiste à s'intéresser à la construction
d'un tel fichier puis de le créer à l'aide de librairies ou bien
directement dans l'IDE. La seconde option est de faire appel à des
logiciels, de préférence multi-plateforme, convertissant un fichier de
nos format de sortie actuel (ex: MIDI) en un fichier au format
souhaité. Cette dernière solution bien qu'algorithmiquement plus
longue à l'avantage d'être codée assez rapidement.


\subsection{Autres fonctionnalités de l'IDE}
Des fonctionnalités suplémentaires ont été imaginées pour
l'IDE. Voici quelques unes des possibilités d'amélioration.

Listen the selection/Écoute de la sélection : L'utilisateur déclenche
la lecture audio de la selection du texte braille en appuyant sur un
bouton ou avec un raccourci clavier.  Cet outil n'a pas pu voir le
jour car elle demande une connaissance approfondie du \textit{braille music}.
En effet il faut pour cela créer une partition correcte avec les
en-têtes nécessaires correspondant uniquement à la partie sélectionné.
Malheureusement nos connaissances dans ce langage sont quasiment
nulles.

%écouter la sélection // déplacer le curseur au fur et à mesure


