%% Organisation du travail
\section{Méthode Scrum}
%% importances des tahces/ périodes de 3 semaines / selection des taches / encadrant pédagogique
Le responsable pédagogique de ce pfa, monsieur Ta, a préféré au classique cahier des charges une utilisation des méthodes agiles, plus précisément de la méthode Scrum.
Cette méthode semble être efficace puisque aujourd'hui, nous avons bien avancé le projet. De plus, elle prépare le groupe à l'entreprise puisque de nos jour, la plupart utilise cette méthode.

\subsection*{Le Product Backlog}
La méthode scrum place tous les collaborateurs du projet sur un pied d'égalité. Il faut tout d'abord créer un Product Backlog qui correspond à toutes les taches à effectuer pour la finalisation du projet.
Ce document doit être validé avec le client, puis ce dernier note les taches par ordre de priorité.

\subsection*{Un Sprint Backlog}
Ces dernières sont ensuites réparties en sprints Backlog. Ce sont des périodes courtes durant lequelles l'équipe travaille sur certaines taches définies précédement. 
À la fin de chaque Srint Backlog l'équipe présente sont travail sous forme concrète au client, sous forme d'un executable, de maquettes ou de toute autre forme.
Ceci est fait pour éviter au projet de partir dans une mauvaise direction et permet au client, dont les envies peuvent changer, de prendre part activement au projet.
La fin d'un sprint laisse place à une nouvelle réunion où sont définit les objectifs du sprint suivant.
Nous avons décidé de faire des Sprint d'une durée approximative de 3 semaines.

\subsection*{Les moyens de communication}
Afin de pouvoir travailer de façon efficace en groupe il est nécessaire de disposer d'outils spécialisés dans la gestion de projet.
Nous avons mis en place trois de ces outils.
Pour faciliter la communication au sein du groupe nous avons créé la liste \textit{pfa.bmc@listes.enseirb-matmeca.fr}. Cette liste est utilisable par tout le monde (membres du groupe, client, résponsable pédagogique) et envoye le mail aux seuls membres du groupe.
Pour gérer la méthode scrum nous communiquons via \textit{Google docs} où nous avons créé une série de fichiers. Les principaux étant le Product Backlog, le détail des Srint Backlog, le compte rendu des réunions avec l'encadrant et un ficher servant de tableau blanc.
Enfin nous avons mis en place un répertoire de travail sur \textit{Github} accécible en écriture aux membres du projet et visible par tous. Celui ci est accécible à l'adresse \textit{https://github.com/ddallago/pfa.bmc}.

