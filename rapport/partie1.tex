%% Organisation du travail
\section{Méthode Scrum}
%% importances des tahces/ périodes de 3 semaines / selection des taches / encadrant pédagogique
L'enseignant pédagogiquede ce pfa, monsieur Ta, a préconisé au groupe de travailler selon les méthodes Agile, et plus précisément en méthode Scrum.
Cette décision étant prise pour nous habituer à ces méthodes que l'on rencontre dans de nombreuses entreprises aujourd'hui.

\subsection*{Le Product Backlog}
La méthode scrum place tous les collaborateurs du projet sur un pied d'égalité. Il faut tout d'abord créer un Product Backlog qui correspond à toutes les taches à effectuer pour la finalisation du projet.
Ce document doit être valider avec le client, puis ce dernier note les taches par ordre de priorité.
\subsection*{Un Sprint Backlog}
Ces dernières sont ensuites réparties en sprint Backlog. Ce sont des periodes courtes durant lequelles l'équipe travaille sur certaines taches définie précédement. 
A la fin de chaque Srint Backlog l'équipe présente sont travail sous forme concrète au client, sous forme d'un executable, de maquettes ou de tout autre forme.
Ceci est fait pour éviter au projet de partir dans une mauvaise direction et permet au client, dont les envies peuvent changer, de prendre part activement au projet.
Nous avons décidé de faire des Sprint de 3 semaines.

\subsection*{}


