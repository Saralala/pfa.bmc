%% Organisation du travail
\section{Méthode Scrum}
%% importances des tahces/ périodes de 3 semaines / selection des taches / encadrant pédagogique
Le responsable pédagogique de ce pfa, monsieur Ta, a préféré au classique cahier des charges une utilisation des méthodes agiles, plus précisément la méthode Scrum.
Cette méthode semble être efficace puisque aujourd'hui, nous avons bien avancé le projet. De plus, elle prépare le groupe à l'entreprise puisqu'aujourd'hui, la plupart utilise cette méthode.

\subsection*{Le Product Backlog}
La méthode scrum place tous les collaborateurs du projet sur un pied d'égalité. Il faut tout d'abord créer un Product Backlog qui correspond à toutes les taches à effectuer pour la finalisation du projet.
Ce document doit être validé avec le client, puis ce dernier note les taches par ordre de priorité.

\subsection*{Un Sprint Backlog}
Ces dernières sont ensuites réparties en sprints Backlog. Ce sont des periodes courtes durant lequelles l'équipe travaille sur certaines taches définies précédement. 
À la fin de chaque Srint Backlog l'équipe présente sont travail sous forme concrète au client, sous forme d'un executable, de maquettes ou de toute autre forme.
Ceci est fait pour éviter au projet de partir dans une mauvaise direction et permet au client, dont les envies peuvent changer, de prendre part activement au projet.
Nous avons décidé de faire des Sprint de 3 semaines.

\subsection*{}


