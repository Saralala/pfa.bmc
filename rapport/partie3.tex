%% Travail effectué
\section{BMC}

\subsection*{}

\subsection*{}

\subsection*{}



\section{Interface graphique}
L'un est critère essentiel de ce projet est l'accessibilité. Par
conséquence pour la réalisation de l'interface graphique, la
bibliothèque GTK+ a été chosie. Cette bibliothèque graphique est bien
connue pour être compatible avec les lecteurs d'écran sur les système
d'exploitation linux.

\subsection*{Forme}
Pour ne pas dépayser le l'utilisateur, l'interface graphique réalisé
possède la forme classique des fenêtres. C'est à dire il présente une
barre de menu listant tous les fonctionnalitées du programme, une
barre d'outils avec les fonctions fréquenment utilisées et une partie
représentant l'activité du programme.  Cette dernière dans le
programme realisé se décompose en deux parties.  Une partie gauche qui
sert à éditer des partitions en braille musique et une partie droite
qui est destinée à afficher ces partions avec les notes visuelles
classiques.



\subsection*{Fonctionnalitées}
Le prototype réalisé comporte tous les fonctionnalitées de base d'un
éditeur de texte. Il permet de créer un nouveau fichier, d'ouvrir un
fichier, d'enregistrer un fichier et d'enregistrer le fichier courant
sous un autre nom. Il permet aussi d'éditer les textes avec les outils
classiques comme copier, couper, coller, séléctionner du texte.
\\ Non seulement, toutes ces fonctions sont accèssible depuis la barre
de menu et/ou la barre d'outils, elles sont également toutes associées
avec des raccourci clavier. Les raccourci utilisés sont très
classiques. Par exemple, Ctrl + o pour ouvrir, Ctrl + c pour copier et
Ctrl + v pour coller.\\ Le prototype possède aussi une procedure de
sécurité qui consiste à vérifier si le fichier en cours d'édition est
sauvegardé ou pas et propose à l'utilisateur de d'enregistrer son
fichier lorsque l'utilisateur tente de créer un nouveau fichier, de
ouvrir un autre fichier ou encore de quitter le programme.




\subsection*{}


\section{BOOST}

\subsection*{}

\subsection*{}

\subsection*
{}
