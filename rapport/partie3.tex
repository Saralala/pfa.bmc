%% Travail effectué
\section{BMC}

\subsection*{}

\subsection*{}

\subsection*{}



\section{Interface graphique}

L'un des critères essentiels de ce projet est l'accessibilité. Par
conséquent, nous avons choisi d'utiliser la bibliothèque GTK+ pour
l'interface graphique sur plateforme linux. En effet, cette
bibliothèque est compatible avec les lecteurs d'écran des systèmes
linux.


\subsection*{Forme}
L'interface graphique est très classique: elle présente une barre de
menu, une barre de raccourcis ainsi que les raccourcis naturels,
encore en cours d'implémentation. La fenêtre se divise en deux
parties: la premiere partie sert pour l'utilisateur malvoyant à éditer les partitions braille. La
deuxième partie sert à l'utilisateur voyant à avoir une partition
classique.


\subsection*{Fonctionnalitées}
Le prototype réalisé comporte tous les fonctionnalitées de base d'un
éditeur de texte. Il permet de créer un nouveau fichier, d'ouvrir un
fichier, d'enregistrer un fichier et d'enregistrer le fichier courant
sous un autre nom. Il permet aussi d'éditer les textes avec les outils
classiques comme copier, couper, coller, séléctionner du texte.
\\ Non seulement, toutes ces fonctions sont accèssible depuis la barre
de menu et/ou la barre d'outils, elles sont également toutes associées
avec des raccourci clavier. Les raccourci utilisés sont très
classiques. Par exemple, Ctrl + o pour ouvrir, Ctrl + c pour copier et
Ctrl + v pour coller.
\\ Le prototype possède aussi une procedure de
sécurité qui consiste à vérifier si le fichier en cours d'édition est
sauvegardé ou non et propose à l'utilisateur d'enregistrer son
fichier lorsque l'utilisateur tente de créer un nouveau fichier,
d'ouvrir un autre fichier ou encore de quitter le programme.




\subsection*{}


\section{BOOST}

\subsection*{}

\subsection*{}

\subsection*
{}
